\documentclass[a4paper,twoside]{article}
\usepackage{blindtext}  
\usepackage{geometry}

% Chinese support
\usepackage[UTF8, scheme = plain]{ctex}

% Page margin layout
\geometry{left=2.3cm,right=2cm,top=2.5cm,bottom=2.0cm}


\usepackage{listings}
\usepackage{xcolor}
\usepackage{geometry}
\usepackage{amsmath}
\usepackage{float}
\usepackage{hyperref}

\usepackage{graphics}
\usepackage{graphicx}
\usepackage{subcaption}
\usepackage{epsfig}
\usepackage{float}

\usepackage{algorithm}
\usepackage[noend]{algpseudocode}

\usepackage{booktabs}
\usepackage{threeparttable}
\usepackage{longtable}
\usepackage{tikz}
\usepackage{multicol}
\usepackage{pgfplots}
\pgfplotsset{compat=1.9}
\pgfplotsset{
    myplotstyle/.style={
    legend style={draw=none, font=\small},
    legend cell align=left,
    legend pos=north east,
    ylabel style={align=center, font=\bfseries\boldmath},
    xlabel style={align=center, font=\bfseries\boldmath},
    x tick label style={font=\bfseries\boldmath},
    y tick label style={font=\bfseries\boldmath},
    scaled ticks=false,
    every axis plot/.append style={thick},
    },
}

% cite package, to clean up citations in the main text. Do not remove.
\usepackage{cite}

\usepackage{color,xcolor}

%% The amssymb package provides various useful mathematical symbols
\usepackage{amssymb}
%% The amsthm package provides extended theorem environments
\usepackage{amsthm}
\usepackage{amsfonts}
\usepackage{enumerate}
\usepackage{enumitem}
\usepackage{listings}
\usepackage{minted}


\usepackage{indentfirst}
\setlength{\parindent}{2em} % Make two letter space in the first paragraph
\usepackage{setspace}
\linespread{1.5} % Line spacing setting
\usepackage{siunitx}
\setlength{\parskip}{0.5em} % Paragraph spacing setting

% \usepackage[contents =22920202204622, scale = 10, color = black, angle = 50, opacity = .10]{background}

\renewcommand{\figurename}{图}
\renewcommand{\listingscaption}{代码}
\renewcommand{\tablename}{表格}
\renewcommand{\contentsname}{目录}
\floatname{algorithm}{算法}

\graphicspath{ {images/} }

%%%%%%%%%%%%%
\newcommand{\StudentNumber}{22920202204622}  % Fill your student number here
\newcommand{\StudentName}{熊恪峥}  % Replace your name here
\newcommand{\PaperTitle}{实验(五)文件系统}  % Change your paper title here
\newcommand{\PaperType}{操作系统实验报告} % Replace the type of your report here
\newcommand{\Date}{2023年6月8日}
\newcommand{\College}{信息学院}
\newcommand{\CourseName}{操作系统}
%%%%%%%%%%%%%

%% Page header and footer setting
\usepackage{fancyhdr}
\usepackage{lastpage}
\pagestyle{fancy}
\fancyhf{}
% This requires the document to be twoside
\fancyhead[LO]{\texttt{\StudentName }}
\fancyhead[LE]{\texttt{\StudentNumber}}
\fancyhead[C]{\texttt{\PaperTitle }}
\fancyhead[R]{\texttt{第{\thepage}页,共\pageref*{LastPage}页}}


\title{\PaperTitle}
\author{\StudentName}
\date{\Date}

\algnewcommand\algorithmicinput{\textbf{Input:}}
\algnewcommand\algorithmicoutput{\textbf{Output:}}
\algnewcommand\Input{\item[\algorithmicinput]}%
\algnewcommand\Output{\item[\algorithmicoutput]}%

\usetikzlibrary{positioning, shapes.geometric,arrows,automata}

\tikzstyle{startstop} = [rectangle, rounded corners, minimum width=3cm, minimum height=1cm, text centered, draw=black, fill=red!30]
\tikzstyle{io} = [trapezium, trapezium left angle=70, trapezium right angle=110, minimum width=3cm, minimum height=1cm, text centered, draw=black, fill=blue!30]
\tikzstyle{process} = [rectangle, minimum width=3cm, minimum height=1cm, text centered, draw=black, fill=green!30]
\tikzstyle{decision} = [diamond, minimum width=3cm, minimum height=1cm, text centered, draw=black, fill=yellow!30]
\tikzstyle{arrow} = [thick,->,>=stealth]

\begin{document}
	
%%%%%%%%%%%%%%%%%%%%%%%%%%%%%%%%%%%%%%%%%%%%
\makeatletter % change default title style
\renewcommand*\maketitle{%
	\begin{center} 
		\bfseries  % title 
		{\LARGE \@title \par}  % LARGE typesetting
		\vskip 1em  %  margin 1em
		{\global\let\author\@empty}  % no author information
		{\global\let\date\@empty}  % no date
		\thispagestyle{empty}   %  empty page style
	\end{center}%
	\setcounter{footnote}{0}%
}
\makeatother
%%%%%%%%%%%%%%%%%%%%%%%%%%%%%%%%%%%%%%%%%%%%
	
	
\thispagestyle{empty}

\vspace*{1cm}

\begin{figure}[htb]
	\centering
	\includegraphics[width=4.0cm]{logo.png}
\end{figure}

\vspace*{1cm}

\begin{center}
	\Huge{\textbf{\PaperType}}
	
	\Large{\PaperTitle}
\end{center}

\vspace*{1cm}

\begin{table}[H]
	\centering	
	\begin{Large}
		\renewcommand{\arraystretch}{1.5}
		\begin{tabular}{p{3cm} p{5cm}<{\centering}}
			姓\qquad 名 & \StudentName  \\
			\hline
			学\qquad号 & \StudentNumber \\
			\hline
			日\qquad期 & \Date  \\
			\hline
			学\qquad院 & \College  \\
			\hline
			课程名称 & \CourseName  \\
			\hline
		\end{tabular}
	\end{Large}
\end{table}

\newpage

\title{
	\Large{\textcolor{black}{\PaperTitle}}
}

\maketitle
	
\tableofcontents
 
\newpage
\setcounter{page}{1}

\begin{spacing}{1.2}

\section{实验目的}

\begin{enumerate}
	\item 学习掌握Linux系统中普通文件和目录文件的区别与联系
	\item 学习掌握Linux管理文件的底层数据结构
	\item 学习掌握Linux文件存储的常见形式
	\item 加深对读写者问题的理解和信号量的使用
\end{enumerate}

\section{实验内容}

\subsection{\texttt{changeDir}函数}

首先使用 findUnitInTable 函数定位目标目录项的索引。
如果返回的索引为-1,则表示该目录项不存在,需要输出提示信息。
接着,检查该目录项是否是目录类型,如果不是需要输出提示信息。
若是目录,则需要修改当前目录,并根据需要修改全局绝对路径。
具体来说,如果目标目录是父目录,则需要去掉路径最后一部分,否则需要在路径和目录名之间进行拼接,以得到新的路径。

\inputminted[firstline=68,lastline=107]{cpp}{../code/File.cpp}

\subsection{\texttt{changeName}函数}

首先需要定位到要修改的目录项。如果文件名不存在,则需要输出提示信息。
如果存在,则可以直接修改目录项的名称。

\inputminted[firstline=109,lastline=119]{cpp}{../code/File.cpp}


\subsection{\texttt{creatDir}函数}

首先需要验证目录名的长度是否合法。
如果目录名过长,需要输出提示信息。
接着,可以使用 getBlock 函数为目录表分配一个大小为 1 的空间。
如果分配失败,则需要输出提示信息。然后需要使用 addDirUnit 函数将新目录作为目录项添加至当前目录。
按照文件系统的规定,还需要向父目录添加一个目录项。
我们可以创建一个新的目录项表,其中需要将目录项数目初始化为 0。
最后需要再次调用 addDirUnit 函数将父目录添加到新目录表,并将父目录设置为当前目录。

\inputminted[firstline=147,lastline=168]{cpp}{../code/File.cpp}


\subsection{\texttt{addDirUnit}函数}

首先需要对目录项的合法性进行检测。
具体分为两个部分:一是检查目录是否已满,二是遍历目录表检查是否存在同名文件。
如果目录项不合法,则添加失败,并显示相应提示信息。
如果目录项合法,则构建新目录项,并设置 fileName 文件名、type 类型和 FCBBlockNum 大小等各个属性。

\inputminted[firstline=200,lastline=224]{cpp}{../code/File.cpp}


\subsection{\texttt{deleteFile}函数}

根据文件系统的规定,自动创建的父目录不能被删除,需要忽略掉。
接着要查找目标文件的索引。
如果查找失败或者查找到的目录项的文件类型是目录,则删除失败,并显示相应提示信息。
否则需要释放该文件的内存并将该目录项从目录表中删除。

\inputminted[firstline=226,lastline=254]{cpp}{../code/File.cpp}


\subsection{\texttt{write\_file}函数}

该函数与 read\_file 函数一起构成读者写者问题。
write\_file 函数比较简单,只需要考虑写者锁,而不需要考虑读者锁。
首先需要获取文件的控制块,接着获取写者锁。如果有写者正在写,则需要等待。
在获取了写者锁之后,就可以开始写操作。首先获取文件大小和要写入数据的大小。
需要注意的是,文件的大小是块的数量乘以块大小的乘积,不能仅将块的数量作为文件大小。
然后使用 getBlockAddr 函数获取写入位置,进行循环写操作。
写操作完成后,需要释放写者锁,以让其他写者写入数据。
此外,需要检测写入的数据大小是否等于文件大小,如果相等,则表示文件已满,输出提示信息。

\inputminted[firstline=412,lastline=436]{cpp}{../code/File.cpp}


\section{实验总结}

本次实验中,我成功地实现了一个类似于 ramfs 的内存文件系统 myRAMFS。通过这个实验,我对文件系统底层存储数据结构的理解得到了加深,并且学习了如何实现一个简单的内存文件系统。在实验中,我通过学习掌握Linux系统中普通文件和目录文件的区别与联系,更深入地理解了Linux管理文件所使用的底层数据结构。通过阅读、理解提供的文件并正确处理文件系统的数据结构,包括目录项、文件节点等,我能够轻松地完成接下来的实验任务。通过类比已经实现的函数,加上注释的指导,我分别实现 changeDir、changeName、creatDir、addDirUnit、deleteFile 和 write\_file 函数。其中 write\_file 函数难度稍大,需要处理读写者问题和信号量的使用,联系了之前学习的内容。最后,我逐步测试了每个功能的正确性。

在实验过程中,我通过实际操作实现了文件系统的创建、删除和文件的写入、读取等基本操作,并且可以支持多个进程同时对文件进行读写操作,更深入地体会到了信号量的使用和读写者问题的解决方法。我们还深入学习了Linux文件存储的常见形式,并且复习了学期初学习的如何使用信号量解决读写者问题。通过本次实验,我不仅巩固了对Linux文件系统底层数据结构的理解,还复习了同步互斥的各种方法,同时增强了对Linux系统编程的能力。总的来说,这个实验不仅对于加深我对文件系统的理解非常有帮助,还提高了我的编程技能和问题解决能力。

\end{spacing}

\end{document}